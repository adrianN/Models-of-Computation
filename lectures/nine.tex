

\subsection{Greedy Embeddings}

We want to embed the graph in a space such that we can successfully find a path between two nodes just by looking at the neighbours and the knowledge about the coordinates. An embedding $f$ is greedy if for any $s\neq t\in V$ then there is a $v\in N(s)$ such that the distance decreases:

\[d(f(v),f(t)) < d(f(s),f(t))\]

In a greedy embedding any $t\in V$ has an edge to $v$ that is closest in the embedding (otherwise $v$ can't be reached greedily).

For example hamiltonian graphs have a greedy embedding on the euclidian plane (draw them as a line). However not all graphs have such an embedding on the plane.

\begin{lem} $K_{k,5k+1}$, $V=A\dot \cup B$ has no greedy embedding in the euclidian plane. 
\end{lem}

\begin{pr} Assume for the sake of contradiction that $f:V\rightarrow \R$ is a greedy embedding.Let $c(v)$ be the closest vertex in A in $f$. There is a vertex $a\in A$ which is the closest neighbour for $b_1,b_2,\ldots b_6\in B$. The circle around $b_{i+1}$ with radius $d(a,b_{i+1})$ must be empty since it can't contain vertices from $A$ by choice of $a$ and it mustn't contain vertices from $B$ (otherwise there would be edges within $B$). 

% pic star

There must be an angle that is less or equal to the average $\leq \pi/3$. Let $\phi_a\leq \pi/3$ be the angle between $b_i$ and $b_{i+1}$. Consider the triangle $a,b_i,b_{i+1}$. Since the anglesum in triangles is $\pi$ there must also be an angle, wlog the largest of those is $\phi_i$, that is $\geq \pi/3$. But then the distance between $b_i$ and $b_{i+1}$ is defined by the law of sines:

\[\sin \phi_a / d(b_i,b_{i+1}) = \sin \phi_i / d(a,b_{i+1})\]

We show $\sin(\phi_i) \geq \sin(\phi_alpha)$. There are two cases. If $\phi_i \leq \pi/2$ we have

\[\sin(\phi_i9 \geq \sin(\phi_\alpha) && \text{as } \pi/2\geq \phi_i\geq \phi_\alpha\]

if on the other hand $\phi_i > \pi/2$ we use $\phi_i < \pi -\phi_alpha$. This holds because of the angle sum in triangles. Thus

\[\sin(\phi_i) > \sin(\pi -\phi_\alpha)\]

because $\sin$ is decreasing in that interval. As $\sin(\pi - \phi_\alpha) = \sin(\phi_\alpha)$, we got that $\sin(\phi_i) > \sin(\phi_alpha)$

Thus from here and from law of sines we get that $d(\alpha, b_i) >= d(\alpha, b_{i+1})$, where equality holds only if we have $\phi_i = \phi_alpha = \pi/3$.
\end{pr}

\begin{thm} Every 3-connected planar graph has a greedy embedding. \qed\end{thm}

The theorem is as good as it can be, since we know $K_{2,11}$ (2-connected) and $K_{3,16}$ (3-connected, not planar) have no embedding.

This is however not useful, since the networks we're considering don't have that property.

\subsection{Greedy routing in normed spaces}

Instead of two dimensions we generalize to $f:V\rightarrow \R^n$ and use a different difference metric. A norm needs to have these properties

\begin{enumerate}
\item $\norm{0} = 0$, $v\neq 0 \Rightarrow \norm{v} >0$
\item $\norm{\alpha v}= |\alpha | \cdot \norm{v}$
\item $\norm{x+y} \leq \norm x +\norm y$
\end{enumerate}

With the norm it is easy to define the distance as the length of the vector between two points.

\begin{lem} Let $(\R^n, \norm{\cdot})$ be a normed space. If it admits a greedy embedding of any graph with $N$ vertices then

\[n \geq \log_3(N-1)\]
\end{lem}

\begin{pr} Consider a star with a central vertex $u$ and $N-1$ vertices around it. Let $f$ be a greedy embedding that maps $u$ to $0$. Let $x_i=f(u_i)$, $l_i=\norm{x_i}$ and $w_i = x_i/l_i$.

Claim: $\norm{w_i-w_j} \geq 1$ if $i\neq j$. Assume otherwise. Consider $d(x_i,x_j)=\norm{x_i-x_j}$.

\begin{align*}
d(x_i,x_j)&=\norm{x_i,x_j} \\
	&=\norm{l_iw_i - l_jw_j}\\
	&=\norm{l_i(w_i-w_j) -(l_j-l_i)w_j}\\
  &\leq l_i \norm{w_i-w_j} + |l_j-l_i|\cdot \norm{w_j} && (l_i\leq l_j)\\
  &< l_i + |l_j-l_i|\\
  &=l_j = \norm{x_j}\\
  &=d(f(u_j),0)
\end{align*}

So the distance between $u_i$ and $u_j$ is strictly smaller than the distance between $u_j$ and the origin, a contradiction.

Hence balls of radius $1/2$ around each $u_i$ is empty and the ball of radius $3/2$ around $u$ contains $N-1$ smaller balls.

\[\Vol(B_{3/2}) \geq (N-1)\Vol(B_{1/2})\]

In normed spaces $\Vol(B_r) = r^n \Vol(B_1)$. We get

\[\frac 32 \Vol(B_1) \geq (N-1)(\frac 12^n \cdot \Vol(B_1)\]

And thus $3^n\geq N-1$
\end{pr}