\section{Routing and Small Worlds}

See \href{http://en.wikipedia.org/wiki/Small_world_experiment}{Wikipedia:Small world experiment} for the general setting.

\subsection{Model and Decentralized Algorithms}

The first observation is that real networks, e.g. social networks, are rich in local connections: we know many people in our university, but few people in neigbouring cities. The model should reflect that. We consider a superposition of two graphs.

Let $G_n(l,f,\alpha)$ be the graph we will consider. The vertex set are points on a $\sqrt{n}\times \sqrt{n}$ grid. Distance $d(u,v)$ between vertices is computed by Manhatten distance. We have two kinds of edges:

\begin{itemize}
\item \emph{Local contacts}: Each vertex is connected to all vertices at distance $\leq l$
\item \emph{Far contacts}: Every vertex has $f$ far contacts, that are randomly chosen. The probability to choose an edge $(u,v)$ is proportional to $d(u,v)$
\[P(\text{$u$ chooses $v$}) = \frac{d(u,v)^{-\alpha}}{\sum_{w\neq u} d(u,w)^{-\alpha}}\]
\end{itemize}

In the pure grid graph with only local contacts the diameter is very high, on the order of $\sqrt n$. Adding random neighbours changes this. The factor $\alpha$ regulates how important the distance is for the probability to choose neighbours. At $\alpha=0$ we have a uniform distribution.

We want to send messages along "short" paths in this graph. We assume that the algorithm knows the local connections of all vertices and the coordinates of the target node $t$. Under certain circumstances we will also assume that we know all the far edges of the previous hops. Let's call such algorithms \emph{decentralised}.

\begin{thm}\label{thm:small_world_graphs} Let $l,f\geq 1$ and let $0\leq \alpha \leq 2$. There is a constant $c=c(l,f,a) > 0$ such that if D denotes the diameter of $G_n(l,f,\alpha)$ then $P(D<c\log n) \geq 1-o(1)$
\end{thm}

Note that if $\alpha \geq 2$ the statement becomes false (the diameter becomes $n^\epsilon$ with $\epsilon > 0$). The next statement concerns finding such paths.

\begin{thm}\label{thm:finding_short_paths} \mbox{}\begin{enumerate}
\item Let $l=f=1$, $\alpha =2$. Then there is a decentralised algorithm $A$ with expected delivery time $O(\log^2n)$
\item Let $l,f\geq 1$, $0 \leq \alpha \lneq 2$. Then there is a $c=c(l,f,\alpha)>0$ such that the expected delivery time of any decentralized algorithm is at least $c n^{\frac{2-\alpha}{3}}$.
\end{enumerate}
\end{thm}

Recall: If $X = \sum_{i=1}^n X_i$
\begin{align*}
Var(X) &= \E\left(\left(\sum X_i\right)^2\right) - \E\left(\sum X_i\right)^2\\
	&= \sum_{1\leq i, j\leq n} (\E(X_iX_j) - \E(X_i)\E(X_j))
\end{align*}

also useful: \href{http://en.wikipedia.org/wiki/Chebyshev\%27s_inequality}{Wikipedia:Chebyshev's inequality}. Further:

\begin{lem} Let $0\leq x \leq 1$. Then
\begin{enumerate}
\item $1-x\leq e^{-x}$
\item $e^{-x} \leq 1-x+\frac{x^2}{2}$
\end{enumerate}
\end{lem}
\begin{pr} Consider the series expansion:
\[e^{-x} = \sum_{i\geq 0} \frac{(-x)^i}{i!} = 1-x+\frac{x^2}{2} - \ldots\]
1. holds because 
\[\sum_{i\geq 1} \frac{x^{2i}}{(2i)!} - \frac{x^{2i+1}}{(2i+1)!} \geq 0\]
and 2. holds because
\[\sum_{i\geq 1} \frac{x^{2i-1}}{(2i-1)!} - \frac{x^{2i}}{(2i)!} \geq 0\]
\end{pr}

\begin{pr}[Theorem \ref{thm:small_world_graphs}] We will assume $\alpha=0$ and $n$ "large enough" to make things easier. Under these assumptions $\forall u: \sum_{v\neq u} d(u,v)^{-\alpha} = n-1$. 

For $S\subset V$ let 
\[X(S) = \bigcup_{v\in S} \tilde N(v) \backslash S\]

where $\tilde N(v)$ is one far-away neighbour $f_v$ of $v$ and $f_v$'s right neighbour. Let $C_0$ be the upper left subgrid of sidelength $\log n$ and let

\[C_{i+1} = X(C_{i-1})\]

We will show that 

\begin{enumerate}
\item with probability $1-o(1/\log n)$ we have $|C_i| \geq 1.5 |C_{i-1}|$, as long as $|C_i| \leq 8 \frac{n}{\log n}$. So after few iterations we see many vertices. 
\item Let $i^*$ be the smallest $i$ s.t. $|C_i| \geq 8\frac{n}{\log n}$. We then proceed to show that every vertex that is not in $C_{i^*}$ can be reachen in O($\log n$) steps.
\end{enumerate}

From these facts follows the theorem, as we can go from any vertex to a vertex in $C_{i^*}$ in O($\log n$) steps, proceed to $C_0$ and take the direct route there ($C_0$ has small diameter).

Let $v\not \in C_{i-1}$. Define a random variable $X_v$ that is $1$ if $v\in C_i$.

\[X_v = \begin{cases} 1& \exists v'\in C_{i-1}: v\in \tilde N(v')\\ 0& \text{otherwise}\end{cases}\]

Clearly the sum of these variables is exactly $|C_i|$. We use Chebishev to bound the size of $C_i$ (so we first compute the variance). Let $|C_i| = S$

Since $X_v$ is an indicator variable we have 

\[\E(X_v) = 1-P(X_v=0)\]
	
$X_v$ is 1 if either $v$ is a far contact of some node, or $v$'s left neighbour, $u$, is such a contact. The probability that one vertex selects $v$ or $u$ is $2/(n-1)$. Since the choices are independent:

\[\E(X_v) = 1-\left(1-\frac{2}{n-1}\right)^S\]

Of course if $v$ is on the left edge of the grid we get $1/(n-1)$, but we'll ignore that for now. For calculating the variances we need $\E(X_uX_v)$ too. We have

\[\E(X_vX_u) = 1-P(X_u=0) - P(X_w=0) + P(X_u=X_v=0)\]

by inclusion-exclusion. The interesting part is $P(X_u=X_v=0)$. There are several cases: If $u$ and $v$ aren't left neighbours of each other and don't lie on the border we mustn't choose one of $4$ nodes. If they lie next to each other there are $3$ forbidden vertices.

\[\E(X_vX_u) = 1-2\left(1-\frac{2}{n-1}\right)^S+\left(1-\frac{\xi}{n-1}\right)^S \quad \xi = \begin{cases}4 &\text{not neighbours}\\ 3 &\text{right neighbours}\end{cases}\]

With these two expectations we can compute the variance. The following formula also considers the edge cases:

\begin{align*}
\E(C_i) &\geq (n-|C_{i-1}| -\sqrt n) \cdot \left(1-\left(1-\frac{2}{n-1}\right)^S\right) \\
 &\geq \left(n- 8\frac{n}{\log n} - \sqrt{n}\right) \cdot \left(1-e^{-\frac{2s}{n-1}}\right)\\
 &\geq \left(n- 9\frac{n}{\log n}\right) \cdot \left(1-e^{-\frac{2s}{n-1}}\right)\\
 &\geq \left(1- 9\frac{n}{\log n}\right)n \cdot \left(1-\left(1-\frac{2s}{n-1}+\frac 12 \left(\frac{2s}{n-1}\right)^2\right)\right)\\
 &\geq \left(1- 9\frac{n}{\log n}\right)\left(2S\left(1+\frac{1}{n-1}-(1+o(1))\frac{2s^2}{n}\right)\right)\\
Since \intertext{$S \in o(n)$}\\
 &\geq (1- o(1))2S\\
\end{align*}
\end{pr}