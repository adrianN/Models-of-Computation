\chapter{Parallel Computing}

\section{Introductary Example}

Given two $n\times n$ matrices $A,B$, we want to find the product.

\[C=AB \qquad (c_{ij} = \sum_{k=1}^n a_{ik} b_{kj})\]

Naive sequential computation takes time $O(n^3)$. The fastest known algorithm takes about $O(n^{2.38})$.

Obviously we can compute the $n^2$ entries in parallel. This gives us a parallel algorithm that takes linear time on $n^2$ processors. We can further increase the parallelism by splitting the sum on $n$ more processors. Combining the results takes only logarithmic time, so we can do matrix multiplication in $O(\log n)$ with $n^3$ processors.

The model we will use for this part of the course is the PRAM, the parallel random access machine. The main difference to the previous topic is the existence of a shared memory that can be accessed by all processors in constant time. We will only consider synchronized versions with a global clock.

There are three types of PRAMS, depending on concurrent memory access is specified:

\begin{itemize}
\item Exclusive Read, Exclusive Write (EREW)
\item Exclusive Write, Concurrent Read (CREW)
\item Concurrent Write, Concurrent Read: What happens there is nondeterministic (other definitions are possible)
\end{itemize}

There are different ways to measure the complexity. One possibility is to model an algorithm as a circuit and use the circuit complexity, ie the total number of gates (the \emph{work} $W$) and the depth $D$ of the circuit . This view abstracts away the actual number of processors available and how to handle memory accesses. 

The running time on $P$ processors is at least $\max(W/P,D)$. 

\begin{thm} An algorithm with work $W$ and depth $D$ can be executed in time $O(W/P+D)$ on a CREW PRAM with $P$ processors.\end{thm}

\begin{pr} Let level $i$ of a circuit be all gates that only use inputs from levels up to $i-1$ and have at least one input from level $i-1$. Level 0 is the input. Let $l_i$ be the number of gates at level $i$

If we proceed levelwise in parallel with $P$ processors we need time

\[\sum_{i=1}^D \rup{\frac{l_i}{P}}\leq \sum_{i=1}^D 1+\frac{l_i}{P} = D+\frac 1P \sum_{i=1}^D l_i = \frac WP +D\]

\end{pr}

\begin{cor} $P= \Omega(W/D)$ processors suffice to achieve running time $O(D)$\end{cor}

\section{Basic parallel algorithms}

\begin{enumerate}
\item given an array A[$1\ldots n$] (w.l.o.g $n=2^k$), we want to find the sum of the entries. 

\begin{lstlisting}
SUM(A) 
if |A| = 1 then
	return A[1]
else
	for $i=1,2\ldots \lceil n/2 \rceil$ do in parallel
		B[i] = A[2i-1] + A[2i]
	return SUM(B)
\end{lstlisting}

\[W(n) = \frac n2 + W(\frac n2) \qquad \rightarrow W(n) = n\]
\[D(n) = 1 + D(\frac n2) \qquad \rightarrow D(n) = \log n\]
\item Prefix sum (/scan): given an array of numbers, find the sum of each prefix of the array $P[i] = \sum_{j\leq i} A[j]$

\begin{lstlisting}
SCAN(A)
if |A|=1 return A
else
	for $i=1,2\ldots \lceil n/2 \rceil$ do in parallel
		B[i] = A[2i-1] + A[2i]
	//S[0] = 0
	S = SCAN(B)
	for $i=1\ldots \lceil n/2 \rceil$ do in parallel
		P[2i] = S[i]
		P[2i-1] = S[i-1] + A[2i-1]
	return P
\end{lstlisting}

\end{enumerate}
