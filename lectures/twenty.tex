\subsection{List ranking}

Recall the list ranking algorithm from section InsertRefHere, that computes the position of each element in the list. The naive algorithm that just traverses the list can incur $O(n)$ IOs if the list doesn't fit in main memory. 

We already saw the parallel solution and want to find an external memory solution.

\paragraph{Phase 1} Scan over the list (in the order in which elements are stored on disk, call this order $L$) and choose a random bit for each active item. Item $w$ generates the triple (pred[v], w, bit of w). Proceed to sort the triples lexicographically, call the resulting sequence $T$. Now scan over $L$ and $T$ in lock-step fashion, i.e., scan item $v$ of $L$ and item (v,w,bit of w) where $v=pred[w]$ at the same time. After the scan $v$ knows its bit and the bit of its successor and can hence decide whether the successor becomes inactive in this round.

Each $v$ sends a message to its predecessor, telling it whether it stays active or becomes inactive.

Items that become inactive send messages to their successors, and stores the current round.

\paragraph{Phase 2} In the second phase we do the same things as in phase 1, except backwards.

\begin{lem} List ranking can be done with $O(\log n \cdot Sort(n))$ IOs. More carefully analysed:

\[\sum_{i=0}^{O(\log n)} \text{Scan}(n) + \text{Sort}(\frac{n}{(3/4)^i})\]
\end{lem}

Let $a=3/4$, then we can write

\begin{align*}
\quad &\sum_{i\geq 0}\text{Sort}(\frac{n}{a^i})\\
=& \sum_{i\geq 0} \frac{n}{aiB}\cdot \frac{\log \frac{n}{a^iB}}{\log (M/B)}\\
\leq & \sum_{i\geq 0} \frac{1}{a^i} \cdot \text{Sort(n)}\\
=& O(\text{Sort}(n))
\end{align*}

We still need to consider the scanning. That can be sped up by compacting the list after each round, i.e. making a copy of the list that only contains active elements while preserving pointers to the original element at each copied element. This can be done in linear time by scanning over the list and determining the number of active elements in front of each element (c.f. the mark and pack algorithm) and assigning these counts as new adresses to each element. In the next step each active item sends a message to its successor "my new number is \ldots". We scan again over the list and make a copy for each active element, link it to its original and set its predecessor value to the message received in the second step. With this technique the number of scanned elements decreases geometrically in each round, such that the total work for scanning is in $O(\text{Sort}(n))$.

Obviously the list ranking algorithm can be implemented in terms of message exchanges. Can this be extended to arbitrary PRAM algorithms?

\section{Simulating a PRAM}
	
Recall, a PRAM has $n$ processors, each with a constant number of registers. In particular each has an adress register -- $R$. It further has a global memory $M$, also of size $n$. In each PRAM step each processor can do some lokal calculations and read or write to $M[R_i]$.

In the external memory model we have just a single CPU. We store $M$ on disk and have an array $P$ that contains the registers of processor $i$ at location $i$. We simulate a step of the PRAM by scanning $P$ and performing the local computations followed by scanning over $P$ again to generate tuples $(R_i, \text{Read/Write}, i, v_i)$ that we then sort. Lastly we scan over the memory and the sorted list of tuples and do the necessary memory updates. For each read at position $j$ we generate a tuple $(i,M[j])$ that we deliver to processor $i$.

\begin{thm} Each CRCW PRAM step takes Sort(n) IOs.\end{thm}