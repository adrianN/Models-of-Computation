The memory layout is similar to the layout in van Emde-Boas trees, with the buffers in between the recursive subparts. 

This structure requires the following space

\begin{align*}
S(K) &= K^{5/2} + (\sqrt K +1)S(\sqrt K)\\
	&= K^{5/2} + (K^{1/2}+1)(K^{5/4} + (K^{1/4}+1)S(K^{1/4}))\\
	&\approx K^{5/2} + K^{7/4} + \underbrace{K^{3/4}(K^{1/4\cdot 5/2}}_{K^{11/8}} + K^{1/8}S(K^{1/8}))\\
	&\leq c\cdot K^{5/2}
	\end{align*}
	
\begin{lem} A $K$-merger requires space $\Theta(K^{5/2})$\end{lem}

\paragraph{A $K$-merger operation} We operate a merger as follows:

\begin{lstlisting}
do K times
	operate each $\sqrt K$ merger at the bottom
	until output buffer is full
	
	operate top $\sqrt K$ merger until either
	$K^2$ outputs are produced or output buffer is full
\end{lstlisting}

The Outermost merger is a $N^{1/3}$-merger. In each of the $N^{1/3}$ rounds it produces $N^{2/3}$ outputs ($N$ in total). Also each input consumes $N^{2/3}$ inputs.

\begin{lem}
If a $J$-merger is invoked $l$ times, it consumes at least $l(J^2)$ inputs from each input line and hence $l\cdot J^3$ inputs.
\end{lem}

\begin{pr} By induction downwards. For $J=N^{1/3}$ this is true as there are never any full buffers. The induction step goes from $J$ to $\sqrt{J}$. 

Each $\sqrt{J}$ merger is invoked $l'$ times, $l'=l\cdot J$, because for each invokation of the $J$-merger, the smaller merger invoked $J$ times. By induction hypothesis, each input line of the $J$-merger consumes $l\cdot J^2$ inputs. So each input line of the bottom $\sqrt J$ mergers consumes at least $l\cdot J^2$ inputs. As $l\cdot J^2 = l\cdot J \cdot (\sqrt{J})^2 = l'\cdot (\sqrt{J})^2$ 
\end{pr}

\paragraph{The cost of the $N^{1/3}$-merger operation} Let $J$ be such that 

\[c\cdot J^{5/2} \leq \frac M4 \leq c\cdot J^5\]

\begin{lem} The $J$-merger plus one cache-line per input buffer fits into cache
\end{lem}

\begin{pr} We need to prove that the input buffers consume $\leq \frac 34 M$. The space for the buffer is

\begin{align*}
\leq J\cdot B &\leq (\frac{M}{4c})^{5/2}\cdot \sqrt{M}\\
	&\leq \frac M2
\end{align*}

because, by tall cache assumption, $B\leq \sqrt M$.
\end{pr}

\begin{lem} The IO cost of one invocation of a $J$-merger is 

\[O(\frac{J^3}{B} + \frac{\text{\# consumed elements}}{B})\]
\end{lem}

\begin{pr} The costs consits of three parts:
\begin{itemize}
\item Moving the $J$-merger into memory: \[\frac{cJ^{5/2}}{B}\]
\item Moving one cache line per input buffer into memory \[J\]
\item Moving the elements processed to the input buffers \[\frac{\text{\# elem}}{B}\]
\end{itemize}

To get the result we need to show $J \leq \frac{J^3}{B}$. By the stronger tall cache assumption $B^{5/2}\leq M$ and we get

\[B^{5/2} \leq M \leq 4cJ^5\]

from which we derive $B\leq (4c)^{2/5} J^2$ and in particular $J=O(J^3/B)$.
\end{pr}

\paragraph{Corollary} For any particular $J$ merger the IO cost is $O(1/B)$ per element processed. 

\paragraph{Proof} If a $J$ merger is invoked $l$ times, it processes at least $J^3\cdot l$ elements. The set-up cost is $l\cdot J^3/B$, i.e. $1/B$ per element processed. The processing cost is the same.

It remains to show through how many $J$ mergers an element flows to get the total IO cost.

\begin{lem} Every element flows through

\[\frac{\log N^{1/3}}{\log J}\]

$J$-mergers
\end{lem}

\begin{pr} Viewed as a binary tree of 2-mergers, a $J$ merger has height $\log J$ and hence an $N^{1/3}$-merger has height $\log N^{1/3}$\end{pr}

\begin{thm} The total IO-cost of moving $N$ elements through an $N^{1/3}$ merger is

\[O(\frac NB \cdot \frac{\log N^{1/3}}{\log J}) = O(\frac NB \cdot \frac{\log (N/B)}{\log (M/B)})\]
\end{thm}

The last equality holds by our assumptions on $J$:

\[J^{5/2} \leq M \leq J^5 \Rightarrow \log J = \Theta(\log M)\]
\[M\geq B^2 \Rightarrow \log M = \Theta(\log (M/B))\]
\[N\geq M \Rightarrow \log N = \Theta(\log (N/B))\]

Hence the total cost for Funnel-Sort is

\[\text{C}(N) = \begin{cases}
N/B & N\leq M\\
N^{1/3} \text{ C}(N^{2/3}) + \frac{N}{B} \cdot \frac{\log (N/B)}{\log (M/B)} & \text{oth.}
\end{cases}\]

The total leaf cost is $N/B$

\begin{align*}
C(N) &= \frac NB \frac{\log (N/B)}{\log (M/B)} + N^{1/3} (\frac{N^{2/3}}{B} \cdot \frac{\log N^{2/3}B}{\log (M/B)} + \ldots\\
\intertext{The $2/3$ in $\log N^{2/3}$ moves out and we see a geometric series, so }
&=O(\frac{1}{B}\cdot \frac{\log (N/B)}{\log (M/B)}
\end{align*}
