\subsection{Colouring Trees}

In order to finde a faster algorithm we first look at a special case, trees. A tree is an acyclic connected graph. All trees are bipartite, and hence have chromatic number 2.

\begin{thm} $T$ is a tree implies $\chi(T)=2$ \end{thm}

\begin{pr} Let $v$ be a distinguished vertex, the root, in the tree. Colour $v$ red.  Colour every other vertex according to the distance from the root. If the distance is odd colour it blue, else red. Since there are no cycles in the graph, the coloring is valid (i.e. it can't happen that a node is on two paths to the root, one of even and one of odd length).
\end{pr}

If we assume that the root knows that it's special we can construct the algorithm in figure \ref{alg:slow_tree_colour}

\begin{figure}[hbt]
\begin{lstlisting}
root: colour with red; send to all children
all except root: 
	wait for message
	if parent red
		colour blue
	else
		color red
	send colour to all children
\end{lstlisting}
\caption{Slow Tree Coloring}
\label{alg:slow_tree_colour}
\end{figure}


This algorithm isn't faster than the previous one, for degenerate trees. We also need to find a way to decide which node is the root. The latter problem is deferred to a later lecture.

Note that this algorithm doesn't need to operate in a synchronous environment. Every node waits for an event before it starts its computation. Event-driven algorithms can be implemented in asynchronous settings

\begin{Def}[Asynchronous DA] An asynchronous distributed algorithm has no access to a global clock and works event-driven.

Any sent message will arrive in finite time.
\end{Def}

There are different ways to assess the complexity of asynchronous algorithms (there are not bounds on the time a message takes). Typically we are interested in 

\begin{itemize}
\item Message Complexity: The number of messages sent
\item Time Complexity: Under the assumption that each message has delay 1, the time complexity is the length of the longest chain of interdependent messages.
\end{itemize}

\begin{thm} The asynchronous time complexity of algorithm \ref{alg:slow_tree_colour} is $\leq \text{height}(T)$ \end{thm}
\begin{pr} In the first round the root sends its colour to all neighbours. This costs one time unit. Argument follows from induction on the height\end{pr}

We now try to improve the algorithm to get a better runtime than $O(n)$. In fact we'll give an algorithm that is bounded by $O(\log^*n)$\footnote{$\log^* n = \min \{i|\log^{(i)} n \leq 2\}$}. We assume that all vertices know that the graph is a tree, a root is known and each node knows it's parent. How we find that information is again deferred to later

\begin{figure}[hbt]
\begin{lstlisting}
Let $c_v$ be ID(v)
repeat 
	send $c_v$ to all children
	if $v$ is the root then
		I=0
	else 
		let $c_p$ be the parent's colour
		I = smallest i s.t. bin$_i$($c_v$), bin$_i$($c_p$) differ
		
	$c_v$ = <bin(I),bin$_I$($c_v$)> //use $1+\log |c_v|$ bits  
											  //even if I small
until |bin($c_v$)| does not change
\end{lstlisting}
\caption{6-colour}
\label{alg:6-colour}
\end{figure}

\begin{lem} Algorithm \ref{alg:6-colour} produces a valid colouring \end{lem}

\begin{pr} Since all IDs are different the colouring is valid in the first step. Now consider some iteration during the execution of the algorithm

Let $u$ be the parent of $w$. If $u$ finds $I_u$ and $w$ finds $I_w$ then the new colours are $c_u=<bin(I_u), bin_{I_u}(c_u)>, c_w=<bin(I_w),b_{I_w}(c_w)>$. There are two cases

\begin{itemize}
\item $I_u\neq I_w$. Then the first part of the new colours differs. \ok
\item $I_u=I_w=I$. Then we have $b_I(c_u)\neq b_I(c_w)$, by the choice of $I$ \ok
\end{itemize}

So the colouring stays valid after each round.
\end{pr}

\begin{lem} Let $K_i$ be the number of bits in the representations of the colours in round $i$.

\begin{align*}
K_0 &= \left\lceil \log n\right\rceil\\
K_{i+1} &= 1+\left\lceil K_i\right\rceil
\end{align*}

\end{lem}

\begin{pr}
We have that $K_{i+1} < K_i$ as long as $K_i\geq 4$. If $K_i\in \{1,2,3\}$, we have $K_{i+1}=K_i$. Otherwise we can write $K_i=2^x\cdot y$, $x\in \N$, $y\in [1,2)$. Then we have

\[K_{i+1} = 1+\left\lceil \log(2^x\cdot y)\right\rceil = \begin{cases} 1+x & y=1\\ 2+x &y>1\end{cases}\]

It follows that

\[K_{i+1}<K_i \Leftrightarrow 2^x\cdot y > \begin{cases} 1+x & y=1\\ 2+x &y>1\end{cases}\]

Check yourself that the exponential function on the left grows sufficiently fast for the claim to be true.
\end{pr}

\begin{thm} The final colouring uses at most 6 colours and stops after $O(\log^* n)$ rounds.\end{thm}

\begin{pr} Let $i$ be the final iteration. Since the algorithm stopped, we know $K_i=K_{i-1} \leq 3$. The colour is encoded as $<bin(I), bin_I(c)>$ since $I\leq 3$ we get only six possible colours.

For the running time we claim: $K_i\leq 2+ \left\lceil \log^{(i)} K_0 \right\rceil$. This can be proven by induction on $i$. 

$i=1$ \ok. $i\rightarrow i+1$:

\begin{align*}
K_{i+1}&= 1+\rup{\log K_i}\\
	&\leq 1+ \rup{\log\left(2+\rup{\log^{(i)}K_0}\right)}\\
	&\leq 1+\rup{1+\log \rup{\log^{(i)} K_0}}\\
	&\leq 2+\rup{1+\log \rup{\log^{(i)} K_0}}\\
	&=\leq 2+\rup{1+\log \log^{(i)} K_0}\\
\end{align*}

If we choose $i=\log^*(K_0)$ we get $K_i\leq 4$ by definition of $\log^*$. Hence in round $\log^*(K_0)+2$ the algorithm will finish.
\end{pr}

Six colours isn't very satisfying for a bipartite graph, so we try to reduce the number of colours further.

\begin{figure}[hbt]
\begin{lstlisting}
root: choose a different colour
all others: take the old colour of parent
\end{lstlisting}
\caption{Shift-Down}
\label{alg:Shift-Down}
\end{figure}

Algorithm \ref{alg:Shift-Down} only takes one round.

\begin{lem} The colouring produced by algorithm \ref{alg:Shift-Down} is valid, if the intitial colouring is valid. Also, all siblings of a node have the same colour.\end{lem}

\begin{pr} Let $c'_u$ be the initial colours and $c_u$ be the new ones. Let $u_1\rightarrow u_2 \rightarrow u_3$ be a path in the tree. Then $c_{u_2}=c'_{u_1} \neq c_{u_3}=c'_{u_2}$. The root is also ok, since it chooses a fresh colour.\end{pr}

\begin{figure}[hbt]
\begin{lstlisting}
for $x\in \{3,4,5\}$
	Shift-Down
	All vertices coloured with $x$ call First-Free
\end{lstlisting}
\caption{6-to-3}
\label{alg:6-to-3}
\end{figure}

\begin{lem} The colouring produced by \ref{alg:6-to-3} is valid and contains $\leq 3$ colours.\end{lem}

\begin{pr} After the shift down, the neighbourhood of a vertex labelled with $x$ contains only two colours. The colour of the parent and the colours of the children. It can also not happen that two adjacent vertices call First-Free simultaniously, since then they would have had the same colour to begin with.\end{pr}

Finding a colouring with just 2 colours is not possible as fast.